% !TEX root = ../om_ts_004.tex

\begin{frame} % frame name
	
	\videotitle{Structural break detection}
	
\end{frame}



\begin{frame}{Structural break detection: Plan}
	\begin{itemize}[<+->]
		\item What is \alert{structural break}?
		\item Detecting \alert{single} structural break
		\item Detecting \alert{several} structural breaks
	\end{itemize}
	
\end{frame}

\begin{frame}
	\frametitle{What is considered a structural break?}
	
	\pause
	Division of the time series into periods between structural break points is \alert{subjective}.
	
	\pause
	Informally, at the moment of structural break  the behavior (pattern) of the series \alert{changes}.
	
	\pause
	What is considered by "changing"?
	
\end{frame}


\begin{frame}
	\frametitle{The idea of detecting a single break}
	
	\begin{itemize}
		\onslide<1->{\item We start with a penalty function that measures \alert{heterogeneity}  in 
			observations $y_a$, $y_{a+1}$, \ldots, $y_b$,
			\[
			C(y_{a:b})
			\]
		}
		\onslide<2->{\item Then we iterate over all moments  $\tau \in [1;T-1]$ and find the minimum of the value
			\[
			C(y_{1:\tau}) + C(y_{\tau+1 : T})
			\]}
		\onslide<3->{We suspect that the break could be  at this point $\tau^*$}
		
		\onslide<4->{\item We assume that the break was in $\tau^*$ if
			the total heterogeneity of the the two fragments is  \alert{significantly}  less than the heterogeneity of the entire series,
			\[
			C(y_{1:\tau^*}) + C(y_{\tau^*+1 : T}) < C(y_{1:T}) - \beta
			\]
		}
	\end{itemize}
	
\end{frame}


\begin{frame}
	\frametitle{Choice of penalty function $C$}
	
	\begin{itemize}
		\onslide<1->{\item There are \alert{many} options}
		\onslide<2->{\item Often we can take the log-likelihood function \alert{of some} model,
			multiplied by minus two:
			\[
			C(y_{a:b}) = -2 \max_{\theta} \ln L(y_a, \ldots, y_b \mid \theta )
			\]
		}
		\onslide<3->{\alert{The simplest model}: $y_t \sim \cN(\mu, \sigma^2)$ and are independent}
		\onslide<4->{\item The choice of the $C$ function is related to the choice of $\beta$ when checking for a break at the assumed break point $\tau^*$,
			\[
			C(y_{1:\tau^*}) + C(y_{\tau^*+1 : T}) < C(y_{1:T}) - \beta
			\]
		}
		\onslide<5->{The more parameters in $\theta$, the larger $\beta$ should be}
	\end{itemize}
	
\end{frame}

\begin{frame}
	\frametitle{How to detect many structural breaks?}
	
	\begin{itemize}
		\onslide<1->{\item Run algorithm to detect \alert{single} structural break}
		
		\onslide<2->{If the algorithm did not detect a break, then we consider that there are no structural break in the series}
		
		\onslide<3->{\item Else  divide the original series into \alert{two sections} according to the detected structural break}
		
		\onslide<4->{\item Then  \alert{recursively} run the detection algorithm for one structural break to \alert{each} of the detected subsections }
	\end{itemize}
	
	
\end{frame}


\begin{frame}
	\frametitle{Transformations before the search}
	
	The structural break can be  \alert{easier} to detect on the transformed series
	
	\begin{itemize}
		\onslide<1->{\item \alert{Simple transformations } of the initial series: logarithm, Box-Cox transformation, transition to differences}
		
		\onslide<2->{\item Decomposition of the series: $STL$, $ETS$, \ldots }
		
	\end{itemize}
	
	
\end{frame}



\begin{frame}
	\frametitle{Why look for structural breaks?}
	
	\begin{itemize}[<+->]
		\item Sometimes structural break detection is the \alert{main goal}
		\item Ability to get \alert{more accurate} forecasts if a dummy variable (equal to one after the break) is added to the set of predictors
		\item Possibility to get \alert{more accurate} forecasts of other series if
		corrected for the structural break series is used as the predictor
	\end{itemize}
	
	
\end{frame}


\begin{frame}{Structural break detection: Summary}
	
	\begin{itemize}[<+->]
		\item There are \alert{many} specialized  algorithms
		\item Does the \alert{sum of inhomogeneities} on the left and right sections to the possible break strongly differ from the heterogeneity of the entire series?
		\item To find \alert{several} breaks, it is enough to search for the next break in the already identified subsections of the series
		\item $STL$ decomposition allows you to search for \alert{breaks in the components} of a series
	\end{itemize}
\end{frame}