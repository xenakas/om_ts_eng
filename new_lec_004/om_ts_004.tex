% arara: xelatex
%% arara: xelatex


% https://koalatea.io/r-knn-regression/
% http://freerangestats.info/blog/2017/04/09/propensity-v-regression
% https://economics.stackexchange.com/questions/45335/what-is-the-difference-between-ate-and-att
% https://kosukeimai.github.io/MatchIt/articles/matching-methods.html


\documentclass[14pt,xcolor=dvipsnames]{beamer}


% !TEX root = om_metrics_14.tex

%\usepackage{epsdice} % dice 1-6 for probability :)

% \usepackage[absolute,overlay]{textpos}

% \usefonttheme[onlymath]{serif}

\usefonttheme{professionalfonts}
% by default beamer changes math fonts for better visibility for projection
% this professionalfonst theme removes this behavior


\usepackage[orientation=portrait,size=custom,width=25.4,height=19.05]{beamerposter}




%25,4 см 19,05 см размеры слайда в powerpoint

\usetheme{metropolis}
\metroset{
  %progressbar=none,
  numbering=none,
  subsectionpage=progressbar,
  block=fill
}

%\usecolortheme{seahorse}

\usepackage{fontspec}
\usepackage{polyglossia}
\setmainlanguage{russian}


% \usepackage{fontawesome5} % removed [fixed]
\setmainfont[Ligatures=TeX]{Myriad Pro}
% \setsansfont{Myriad Pro}




% why do we need \newfontfamily:
% http://tex.stackexchange.com/questions/91507/
\newfontfamily{\cyrillicfonttt}{Myriad Pro}
\newfontfamily{\cyrillicfont}{Myriad Pro}
%\newfontfamily{\cyrillicfontbs}{Myriad Pro}
\newfontfamily{\cyrillicfontsf}{Myriad Pro}


% https://tex.stackexchange.com/questions/175860/why-does-unicode-math-break-the-kerning-of-accents-in-combination-with-amssymb
% "You shouldn't be using amssymb together with unicode-math"
\usepackage{amsmath}
\usepackage{amsthm} % amssymb 


% https://tex.stackexchange.com/questions/483722/
% \usepackage[MnSymbol]{mathspec}  % Includes amsmath.
% \usepackage{mathspec}  % Includes amsmath.
% \setmathsfont(Digits,Latin,Greek,Symbols)[Numbers={Lining,Proportional}]{Latin Modern Math}
% mathspec must be loaded earlier than amsmath



%\usepackage{bm}

% \usepackage{fdsymbol} % \nperp

% \usepackage{unicode-math} % \symbf
% \setmathfont{Latin Modern Math}



\usepackage{centernot}

\usepackage{graphicx}

\usepackage{wrapfig}
% \usepackage{animate} % animations :)
% \usepackage{tikz}
%\usetikzlibrary{shapes.geometric,patterns,positioning,matrix,calc,arrows,shapes,fit,decorations,decorations.pathmorphing}
% \usepackage{pifont}
\usepackage{comment}
\usepackage[font=small,labelfont=bf]{caption}
\captionsetup[figure]{labelformat=empty}
% \includecomment{techno}



%Расположение

\setbeamersize{text margin left=15 mm,text margin right=5mm} 
\setlength{\leftmargini}{38 pt}

%\usepackage{showframe}
%\usepackage{enumitem}
% \setlist{leftmargin=5.5mm}


%Цвета от дирекции

\definecolor{dirblack}{RGB}{58, 58, 58}
\definecolor{dirwhite}{RGB}{245, 245, 245}
\definecolor{dirred}{RGB}{149, 55, 53}
\definecolor{dirblue}{RGB}{0, 90, 171}
\definecolor{dirorange}{RGB}{235, 143, 76}
\definecolor{dirlightblue}{RGB}{75, 172, 198}
\definecolor{dirgreen}{RGB}{155, 187, 89}
\definecolor{dircomment}{RGB}{128, 100, 162}

\setbeamercolor{title separator}{bg=dirlightblue!50, fg=dirblue}

%Цвета блоков

% Голубой блок!
\setbeamercolor{block title}{bg=dirblue!30,fg=dirblack}
\setbeamercolor{block title example}{bg=dirlightblue!50,fg=dirblack}
\setbeamercolor{block body example}{bg=dirlightblue!20,fg=dirblack}

\AtBeginEnvironment{exampleblock}{\setbeamercolor{itemize item}{fg=dirblack}}
%\setbeamertemplate{blocks}[rounded][shadow]

% Набор команд для удобства верстки

% Набор команд для структуризации

%\newcommand{\quest}{\faQuestionCircleO}
%\faPencilSquareO \faPuzzlePiece \faQuestionCircleO  \faIcon*[regular]{file} {\textcolor{dirblue}
%\newcommand{\quest}{\textcolor{dirblue}{\boxed{\textbf{?}}}
%\newcommand{\task}{\faIcon{tasks}}
%\newcommand{\exmpl}{\faPuzzlePiece}
%\newcommand{\dfn}{\faIcon{pen-square}}
%\newcommand{\quest}{\textcolor{dirblue}{\faQuestionCircle[regular]}}
%\newcommand{\acc}[1]{\textcolor{dirred}{#1}}
%\newcommand{\accm}[1]{\textcolor{dirred}{#1}}
%\newcommand{\acct}[1]{\textcolor{dirblue}{#1}}
%\newcommand{\acctm}[1]{\textcolor{dirblue}{#1}}
%\newcommand{\accex}[1]{\textcolor{dirblack}{\bf #1}}
%\newcommand{\accexm}[1]{\textcolor{dirblack}{ \mathbf{#1}}}
%\newcommand{\acclp}[1]{\textcolor{dirorange}{\it #1}}
\newcommand{\todo}[1]{\textcolor{dircomment}{\bf #1}}
%\newcommand{\graylink}[1]{{\fontsize{11}{12}\selectfont \textcolor{gray}{#1}}}
%\newcommand{\figcaption}[1]{{\fontsize{18}{20}\selectfont #1}}


\newcommand{\videotitle}[1]{
    {\fontsize{33}{30}\selectfont \textcolor{dirblue}{\textbf{#1}} }

    %\todo{название видеофрагмента}
}

\newcommand{\lecturetitle}[1]{
  {\fontsize{33}{30}\selectfont \textcolor{dirblue}{\textbf{#1}} }

    %\todo{название лекции}
}





%\newcommand{\spcbig}{\vspace{-10 pt}}
%\newcommand{\spcsmall}{\vspace{-5 pt}}

%\usepackage{listings}
%\lstset{
%xleftmargin=0 pt,
%  basicstyle=\small, 
%  language=Python,
  %tabsize = 2,
%  backgroundcolor=\color{mc!20!white}
%}



%\newcommand{\mypart}[1]{\begin{frame}[standout]{\huge #1}\end{frame}}

\setbeamercolor{background canvas}{bg=}

% frame title setup
\setbeamercolor{frametitle}{bg=,fg=dirblue}
\setbeamertemplate{frametitle}[default][left]

\addtobeamertemplate{frametitle}{\hspace*{0.1 cm}}{\vspace*{0.25cm}}


%Шрифты
\setbeamerfont{frametitle}{family=\rmfamily,series=\bfseries,size={\fontsize{33}{30}}}
\setbeamerfont{framesubtitle}{family=\rmfamily,series=\bfseries,size={\fontsize{26}{20}}}


% удобнее знать номер слайда, чтобы вносить правки!  

\setbeamercolor{footline}{fg=dircomment}
\setbeamerfont{footline}{series=\bfseries, size={\fontsize{12}{14}}}
%\setbeamertemplate{footline}[page number]


\defbeamertemplate{footline}{custom footline}
{%
  \hspace*{\fill}%
  \usebeamercolor[fg]{page number in head/foot}%
  \usebeamerfont{page number in head/foot}%
  page: \insertpagenumber\,/\,\insertpresentationendpage%
  \hspace{20pt}%
  slide: \insertframenumber\,/\,\inserttotalframenumber%
  %\hspace*{\fill}
  \vskip2pt%
}
%\setbeamertemplate{footline}[custom footline]

\usepackage{physics}
\usepackage[makeroom]{cancel}



% tikz block

\usepackage{pgfplots}
\pgfplotsset{compat=newest}

\usepackage{tikz}
\usetikzlibrary{calc}
\usetikzlibrary{quotes,angles}
\usetikzlibrary{arrows}
\usetikzlibrary{arrows.meta}
\usetikzlibrary{positioning,intersections,decorations.markings}
\usetikzlibrary{patterns}

\usepackage{tkz-euclide} 
%\tikzset{>=latex}

\tikzset{cross/.style={cross out, draw=black, minimum size=2*(#1-\pgflinewidth), inner sep=0pt, outer sep=0pt},
%default radius will be 1pt. 
cross/.default={5pt}}

\colorlet{veca}{red}
\colorlet{vecb}{blue}
\colorlet{vecc}{olive}


\newcommand{\grid}{\draw[color=gray,step=1.0,dotted] (-2.1,-2.1) grid (9.6,6.1)}

% end tikz block

\newcommand{\R}{\mathbb{R}}
\newcommand{\Rot}{\mathrm{R}}
\newcommand{\HH}{\mathrm{H}}
\newcommand{\Id}{\mathrm{I}}
\newcommand{\RR}{\mathbb{R}}
\newcommand{\ZZ}{\mathbb{Z}}
\newcommand{\la}{\lambda}
\let\P\relax
\newcommand{\P}{\mathbb{P}}
\newcommand{\E}{\mathbb{E}}

\newcommand{\cN}{\mathcal{N}}
\newcommand{\dN}{\mathcal{N}}

\newcommand{\qL}{q_{\text{left}}}
\newcommand{\qR}{q_{\text{right}}}



\newcommand{\ba}{\mathbf{a}}
\newcommand{\be}{\mathbf{e}}
\newcommand{\bb}{\mathbf{b}}
\newcommand{\bc}{\mathbf{c}}
\newcommand{\bd}{\mathbf{d}}
\newcommand{\bx}{\mathbf{x}}
\newcommand{\bff}{\mathbf{f}} % \bf is already def
\newcommand{\bv}{\mathbf{v}}
\newcommand{\bzero}{\mathbf{0}}



\DeclareMathOperator{\Var}{Var}
\DeclareMathOperator{\sVar}{sVar}
\DeclareMathOperator{\Cov}{Cov}
\DeclareMathOperator{\sCov}{sCov}
\DeclareMathOperator{\sCorr}{sCorr}
\DeclareMathOperator{\pCorr}{pCorr}
\DeclareMathOperator{\Corr}{Corr}
\DeclareMathOperator{\Med}{Med}
\let\L\relax
\DeclareMathOperator{\L}{L}


\DeclareMathOperator{\plim}{plim}
\DeclareMathOperator{\sign}{sign}


\newcommand{\graylink}[1]{{\fontsize{11}{12}\selectfont \textcolor{gray}{#1}}}
\newcommand{\figcaption}[1]{{\fontsize{18}{20}\selectfont #1}}





\begin{document}


\begin{frame} % название лекции


\lecturetitle{Pre-processing data}

\end{frame}


% !TEX root = ../om_ts_004.tex

\begin{frame} % frame name
	
	\videotitle{Handling missing data}
	
\end{frame}



\begin{frame}{Handling missing data: Plan}
	\begin{itemize}[<+->]
		\item Linear \alert{interpolation}
		\item \alert{Modelling} approach
		\item Using \alert{STL decomposition}
	\end{itemize}
	
\end{frame}


\begin{frame}
	\frametitle{Linear Interpolation}
	
	\begin{block}{Idea}
		To  fill in missing data  we need to the restore the values so that they fit perfectly on the straight line (form an \alert{arithmetic progression}),
		\[
		\Delta y_t^{imp} = const
		\]
	\end{block}
	\pause
	
	Example:
	
	10, \alert{NA}, \alert{NA}, 100
	
	\pause
	
	10, \alert{40}, \alert{70}, 100
	
\end{frame}


\begin{frame}
	\frametitle{Modelling approach to handle the missing data}
	
	\begin{enumerate}[<+->]
		\item Evaluate a model that \alert{allows} the missing data:
		
		ARIMA or automatic ARIMA works fine!

		\item Missing values of $y_t$ can be replaced by the conditional mathematical expectation,
		assuming the estimated parameters of the model to be true,
		\[
		y_t^{imp} = \E(y_t \mid \text{data})
		\]
		For that \alert{Kalman filter} is used
	\end{enumerate}
	
	\pause
	The ability to evaluate a model on data with missing values is highly dependent on \alert{implementation}
	
\end{frame}

\begin{frame}
	\frametitle{Using STL decomposition}
	
	\begin{enumerate}[<+->]
		\item We decompose the series with missing data into components:
		\[
		y_t = \text{trend}_t + \text{seasonal}_t + \text{remainder}_t = \text{seasonal}_t + \text{deseason}_t.
		\]	
		STL restores the \alert{seasonal component} without the gaps
		!
		\item Recover the missing values of the deseasoned series by \alert{linear} interpolation
		
		\item The missing values of $y_t$ are replaced by the sum of the restored deseasoned values and the seasonal component,
		\[
		y_t^{imp} = \text{seasonal}_t + \text{deseason}_t^{imp}
		\]
		
	\end{enumerate}
	
\end{frame}

\begin{frame}
	\frametitle{Why we need to handle the missing data?}
	
	\begin{itemize}[<+->]
		\item Filling in the gaps is sometimes a \alert{main task}
		\item Possibility to use \alert{more algorithms} of prediction for the reconstructed series
		\item Ability to use the restored row \alert{as a predictor}
	\end{itemize}
	
	
\end{frame}


\begin{frame}{Handling missing data: Summary}
	
	\begin{itemize}[<+->]
		\item Linear \alert{interpolation}: simple and fast!
		\item Using \alert{ARIMA} or more complex models
		\item \alert{STL decomposition} and restoring components
		\item \alert{Variations} for each algorithm
	\end{itemize}
\end{frame}

% !TEX root = ../om_ts_004.tex

\begin{frame} % frame name
	
	\videotitle{Anomaly detection}
	
\end{frame}



\begin{frame}{Anomaly detection: Plan}
	\begin{itemize}[<+->]
		\item Which observation is \alert{anomalous}?
		\item Algorithms for \alert{detection and correction} of anomalies
		\item \alert{Why} we look for anomalies?
	\end{itemize}
	
\end{frame}


\begin{frame}
	\frametitle{Which observation is considered anomalous?}
	
	
	\pause
	Categorizing the observations into anomalous and ordinary one is \alert{subjective}.
	
	\pause
	Informally, an anomalous observation \alert{doesn't fit} into the \alert{normal dynamics} of the series.
	
	\pause
	What is considered a "normal dynamics"? What does "not fitting in" mean?
	
\end{frame}

\begin{frame}
	\frametitle{Anomaly detection algorithm}
	
	\begin{itemize}
		\onslide<1->{\item We take any algorithm that allows us to obtain   \alert{residuals}  $\hat u_t$ from the series}

		\onslide<2->{Residuals is the difference between the actual value and the prediction within the training set}
		

		\onslide<3->{The $ARIMA$, $ETS$, \ldots models, as well as the $STL$ algorithm will do}
		
		
		\onslide<4->{\item Estimate \alert{standard error} of the residuals}
		
		\onslide<5->{\item If the absolute value of the remainder is greater than \alert{three} standard errors, we consider the observation to be anomalous}
		
	\end{itemize}
	
\end{frame}


\begin{frame}
	\frametitle{Correction of anomalies}
	
	\pause
	We subtract the remainder from the anomalous observation:
	\[
	y_t^{imp} = y_t - \hat u_t
	\]
	
\end{frame}



\begin{frame}
	\frametitle{Why look for anomalous sightings?}
	
	\begin{itemize}[<+->]
		\item Sometimes anomaly detection is the \alert{main goal}
		\item Possibility to get \alert{more accurate} predictions for the corrected series
		\item Possibility to get \alert{more accurate} predictions by using the corrected series as a predictor
	\end{itemize}
	
	
\end{frame}



\begin{frame}{Anomaly detection: Summary}
	
	\begin{itemize}[<+->]
		\item We take any algorithm (STL, ARIMA, ETS, \ldots) that extracts \alert{residual} from the series
		\item There are \alert{a bunch} of special algorithms
		\item If the remainder \alert{is large}, then we consider the observation to be anomalous
		\item To correct the anomalous observation, we replace the its remainder with  \alert{zero}
		\item \alert{Correction} of anomalous observations before forecasting can improve forecasts!
	\end{itemize}
\end{frame}

% !TEX root = ../om_ts_004.tex

\begin{frame} % frame name
	
	\videotitle{Structural break detection}
	
\end{frame}



\begin{frame}{Structural break detection: Plan}
	\begin{itemize}[<+->]
		\item What is \alert{structural break}?
		\item Detecting \alert{single} structural break
		\item Detecting \alert{several} structural breaks
	\end{itemize}
	
\end{frame}

\begin{frame}
	\frametitle{What is considered a structural break?}
	
	\pause
	Division of the time series into periods between structural break points is \alert{subjective}.
	
	\pause
	Informally, at the moment of structural break  the behavior (pattern) of the series \alert{changes}.
	
	\pause
	What is considered by "changing"?
	
\end{frame}


\begin{frame}
	\frametitle{The idea of detecting a single break}
	
	\begin{itemize}
		\onslide<1->{\item We start with a penalty function that measures \alert{heterogeneity}  in 
			observations $y_a$, $y_{a+1}$, \ldots, $y_b$,
			\[
			C(y_{a:b})
			\]
		}
		\onslide<2->{\item Then we iterate over all moments  $\tau \in [1;T-1]$ and find the minimum of the value
			\[
			C(y_{1:\tau}) + C(y_{\tau+1 : T})
			\]}
		\onslide<3->{We suspect that the break could be  at this point $\tau^*$}
		
		\onslide<4->{\item We assume that the break was in $\tau^*$ if
			the total heterogeneity of the the two fragments is  \alert{significantly}  less than the heterogeneity of the entire series,
			\[
			C(y_{1:\tau^*}) + C(y_{\tau^*+1 : T}) < C(y_{1:T}) - \beta
			\]
		}
	\end{itemize}
	
\end{frame}


\begin{frame}
	\frametitle{Choice of penalty function $C$}
	
	\begin{itemize}
		\onslide<1->{\item There are \alert{many} options}
		\onslide<2->{\item Often we can take the log-likelihood function \alert{of some} model,
			multiplied by minus two:
			\[
			C(y_{a:b}) = -2 \max_{\theta} \ln L(y_a, \ldots, y_b \mid \theta )
			\]
		}
		\onslide<3->{\alert{The simplest model}: $y_t \sim \cN(\mu, \sigma^2)$ and are independent}
		\onslide<4->{\item The choice of the $C$ function is related to the choice of $\beta$ when checking for a break at the assumed break point $\tau^*$,
			\[
			C(y_{1:\tau^*}) + C(y_{\tau^*+1 : T}) < C(y_{1:T}) - \beta
			\]
		}
		\onslide<5->{The more parameters in $\theta$, the larger $\beta$ should be}
	\end{itemize}
	
\end{frame}

\begin{frame}
	\frametitle{How to detect many structural breaks?}
	
	\begin{itemize}
		\onslide<1->{\item Run algorithm to detect \alert{single} structural break}
		
		\onslide<2->{If the algorithm did not detect a break, then we consider that there are no structural break in the series}
		
		\onslide<3->{\item Else  divide the original series into \alert{two sections} according to the detected structural break}
		
		\onslide<4->{\item Then  \alert{recursively} run the detection algorithm for one structural break to \alert{each} of the detected subsections }
	\end{itemize}
	
	
\end{frame}


\begin{frame}
	\frametitle{Transformations before the search}
	
	The structural break can be  \alert{easier} to detect on the transformed series
	
	\begin{itemize}
		\onslide<1->{\item \alert{Simple transformations } of the initial series: logarithm, Box-Cox transformation, transition to differences}
		
		\onslide<2->{\item Decomposition of the series: $STL$, $ETS$, \ldots }
		
	\end{itemize}
	
	
\end{frame}



\begin{frame}
	\frametitle{Why look for structural breaks?}
	
	\begin{itemize}[<+->]
		\item Sometimes structural break detection is the \alert{main goal}
		\item Ability to get \alert{more accurate} forecasts if a dummy variable (equal to one after the break) is added to the set of predictors
		\item Possibility to get \alert{more accurate} forecasts of other series if
		corrected for the structural break series is used as the predictor
	\end{itemize}
	
	
\end{frame}


\begin{frame}{Structural break detection: Summary}
	
	\begin{itemize}[<+->]
		\item There are \alert{many} specialized  algorithms
		\item Does the \alert{sum of inhomogeneities} on the left and right sections to the possible break strongly differ from the heterogeneity of the entire series?
		\item To find \alert{several} breaks, it is enough to search for the next break in the already identified subsections of the series
		\item $STL$ decomposition allows you to search for \alert{breaks in the components} of a series
	\end{itemize}
\end{frame}


\end{document}
