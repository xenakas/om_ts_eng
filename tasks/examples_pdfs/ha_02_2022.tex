% arara: xelatex
%% arara: xelatex

\documentclass[12pt]{article}

\usepackage{tikz} % картинки в tikz
\usepackage{microtype} % свешивание пунктуации

\usepackage{array} % для столбцов фиксированной ширины

\usepackage{indentfirst} % отступ в первом параграфе

\usepackage{sectsty} % для центрирования названий частей
\allsectionsfont{\centering}

\usepackage{amsmath, amssymb, amsthm} % куча стандартных математических плюшек

\usepackage{amsfonts}

\usepackage{comment}

\usepackage[top=2cm, left=1.2cm, right=1.2cm, bottom=2cm]{geometry} % размер текста на странице

\usepackage{lastpage} % чтобы узнать номер последней страницы

\usepackage{enumitem} % дополнительные плюшки для списков
%  например \begin{enumerate}[resume] позволяет продолжить нумерацию в новом списке
\usepackage{caption}


\usepackage{hyperref} % гиперссылки

\usepackage{multicol} % текст в несколько столбцов


\usepackage{fancyhdr} % весёлые колонтитулы
\pagestyle{fancy}
\lhead{Анализ временных рядов, Экономический анализ, НИУ-ВШЭ}
\rhead{Проект}

\cfoot{}
\rfoot{}
% \rfoot{\thepage/3}
\renewcommand{\headrulewidth}{0.4pt}
\renewcommand{\footrulewidth}{0.4pt}



% \usepackage{todonotes} % для вставки в документ заметок о том, что осталось сделать
% \todo{Здесь надо коэффициенты исправить}
% \missingfigure{Здесь будет Последний день Помпеи}
% \listoftodos - печатает все поставленные \todo'шки


% более красивые таблицы
\usepackage{booktabs}
% заповеди из документации:
% 1. Не используйте вертикальные линии
% 2. Не используйте двойные линии
% 3. Единицы измерения - в шапку таблицы
% 4. Не сокращайте .1 вместо 0.1
% 5. Повторяющееся значение повторяйте, а не говорите "то же"



\usepackage{fontspec}
\usepackage{polyglossia}

\setmainlanguage{russian}
\setotherlanguages{english}

% download "Linux Libertine" fonts:
% http://www.linuxlibertine.org/index.php?id=91&L=1
\setmainfont{Linux Libertine O} % or Helvetica, Arial, Cambria
% why do we need \newfontfamily:
% http://tex.stackexchange.com/questions/91507/
\newfontfamily{\cyrillicfonttt}{Linux Libertine O}

\AddEnumerateCounter{\asbuk}{\russian@alph}{щ} % для списков с русскими буквами
\setlist[enumerate, 2]{label=\asbuk*),ref=\asbuk*}

%% эконометрические сокращения
\let\P\relax
\DeclareMathOperator{\P}{\mathbb{P}}
\DeclareMathOperator{\Cov}{\mathbb{C}ov}
\DeclareMathOperator{\Corr}{\mathbb{C}orr}
\DeclareMathOperator{\Var}{\mathbb{V}ar}
\DeclareMathOperator{\E}{\mathbb{E}}
\DeclareMathOperator{\tr}{trace}
\newcommand \hb{\hat{\beta}}
\newcommand \hs{\hat{\sigma}}
\newcommand \htheta{\hat{\theta}}
\newcommand \s{\sigma}
\newcommand \hy{\hat{y}}
\newcommand \hY{\hat{Y}}
\newcommand \vone{\vec{1}}
\newcommand \e{\varepsilon}
\newcommand \he{\hat{\e}}
\newcommand \z{z}
\newcommand \hVar{\widehat{\Var}}
\newcommand \hCorr{\widehat{\Corr}}
\newcommand \hCov{\widehat{\Cov}}
\newcommand \cN{\mathcal{N}}



\begin{document}

\section*{Проект}

\begin{enumerate}

\item Возьмите любой ряд с ежемесячными наблюдениями. 

\item (30 баллов) Визуализируйте сам ряд, ряд обычных и сезонных разностей, компоненты ряда,
обычные и частные автокорреляционные функции. 

Прокомментируйте графики. 

\item (10 баллов) Является ряд стационарным? 

Подтвердите ответ подходящим тестом. 

\item (10 баллов) Если разумно применить к исходному ряду какое-либо преобразование, 
то примените его, мотивировав свой выбор. 

\item (5 баллов) Поделите ряд на тестовую и обучающую выборку. 

\item (60 баллов) Оцените ряд моделей/алгоритмов на тестовой выборке. 

Здесь вы ограничены только вашей фантазией!
Как минимум следует рассмотреть: наивную модель, ETS, SARIMA, случайный лес, 
тета-метод и усреднение моделей-лидеров.
Можно в качестве предикторов взять дополнительные ряды. 
Можно прогнозировать компоненты ряда разными моделями. 

\item (10 баллов) Выберите наилучшую модель. 

Визуализируйте ряд остатков наилучшей модели на обучающей выборке, прокомментируйте.
Постройте график прогнозов на два года вперед, переоценив наилучшую моделей по полной выборке.
Не забудьте вернуться к исходному ряду, если вы делали преобразование.


\item (20 баллов) Удивите проверяющих реализацией какой-нибудь интересной дополнительной идеи. 

Можно заполнить пропуски, проверить наличие структурных сдвигов, выявить аномальные наблюдения, 
можно использовать техники не упомянутые в курсе. 

\item (бонус, 10 баллов) Если вы занимались прогнозированием рядов вне данного курса,
кратко опишите проблемы с которыми вам пришлось столкнуться и как вы их решали. 
Если не занимались, то расскажите, какие сюжеты курса дались легко, а какие вызывают сложности. 

    
\end{enumerate}


\section*{Комментарии}

Работу следует представить в виде отчёта в pdf или html формате. 
В начале работы должен идти текст с графиками, в конце работы в качестве приложения должен идти код. 
Общий объем текста (без приложений) должен составлять \textbf{не более 10 страниц}.

В отличие от домашнего задания формулировка проекта во многих пунктах является вольной и предполагает творческий подход. 
В каждом пункте можно сделать больше, чем указанный минимум.

Проект можно выполнять в одиночку, а также группой из двух или трёх человек. 

Исходные данные должны быть прикреплены к работе или загружаться автоматически из открытых источников в интернете.

Дедлайн сдачи - \textbf{28 марта 2022, 20:59}. До указанного времени файл должен быть загружен по ссылке
\url{}.







\end{document}


