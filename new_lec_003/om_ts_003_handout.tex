% arara: xelatex
%% arara: xelatex


% https://koalatea.io/r-knn-regression/
% http://freerangestats.info/blog/2017/04/09/propensity-v-regression
% https://economics.stackexchange.com/questions/45335/what-is-the-difference-between-ate-and-att
% https://kosukeimai.github.io/MatchIt/articles/matching-methods.html


\documentclass[14pt,xcolor=dvipsnames,handout]{beamer}


% !TEX root = om_metrics_14.tex

%\usepackage{epsdice} % dice 1-6 for probability :)

% \usepackage[absolute,overlay]{textpos}

% \usefonttheme[onlymath]{serif}

\usefonttheme{professionalfonts}
% by default beamer changes math fonts for better visibility for projection
% this professionalfonst theme removes this behavior


\usepackage[orientation=portrait,size=custom,width=25.4,height=19.05]{beamerposter}




%25,4 см 19,05 см размеры слайда в powerpoint

\usetheme{metropolis}
\metroset{
  %progressbar=none,
  numbering=none,
  subsectionpage=progressbar,
  block=fill
}

%\usecolortheme{seahorse}

\usepackage{fontspec}
\usepackage{polyglossia}
\setmainlanguage{russian}


% \usepackage{fontawesome5} % removed [fixed]
\setmainfont[Ligatures=TeX]{Myriad Pro}
% \setsansfont{Myriad Pro}




% why do we need \newfontfamily:
% http://tex.stackexchange.com/questions/91507/
\newfontfamily{\cyrillicfonttt}{Myriad Pro}
\newfontfamily{\cyrillicfont}{Myriad Pro}
%\newfontfamily{\cyrillicfontbs}{Myriad Pro}
\newfontfamily{\cyrillicfontsf}{Myriad Pro}


% https://tex.stackexchange.com/questions/175860/why-does-unicode-math-break-the-kerning-of-accents-in-combination-with-amssymb
% "You shouldn't be using amssymb together with unicode-math"
\usepackage{amsmath}
\usepackage{amsthm} % amssymb 


% https://tex.stackexchange.com/questions/483722/
% \usepackage[MnSymbol]{mathspec}  % Includes amsmath.
% \usepackage{mathspec}  % Includes amsmath.
% \setmathsfont(Digits,Latin,Greek,Symbols)[Numbers={Lining,Proportional}]{Latin Modern Math}
% mathspec must be loaded earlier than amsmath



%\usepackage{bm}

% \usepackage{fdsymbol} % \nperp

% \usepackage{unicode-math} % \symbf
% \setmathfont{Latin Modern Math}



\usepackage{centernot}

\usepackage{graphicx}

\usepackage{wrapfig}
% \usepackage{animate} % animations :)
% \usepackage{tikz}
%\usetikzlibrary{shapes.geometric,patterns,positioning,matrix,calc,arrows,shapes,fit,decorations,decorations.pathmorphing}
% \usepackage{pifont}
\usepackage{comment}
\usepackage[font=small,labelfont=bf]{caption}
\captionsetup[figure]{labelformat=empty}
% \includecomment{techno}



%Расположение

\setbeamersize{text margin left=15 mm,text margin right=5mm} 
\setlength{\leftmargini}{38 pt}

%\usepackage{showframe}
%\usepackage{enumitem}
% \setlist{leftmargin=5.5mm}


%Цвета от дирекции

\definecolor{dirblack}{RGB}{58, 58, 58}
\definecolor{dirwhite}{RGB}{245, 245, 245}
\definecolor{dirred}{RGB}{149, 55, 53}
\definecolor{dirblue}{RGB}{0, 90, 171}
\definecolor{dirorange}{RGB}{235, 143, 76}
\definecolor{dirlightblue}{RGB}{75, 172, 198}
\definecolor{dirgreen}{RGB}{155, 187, 89}
\definecolor{dircomment}{RGB}{128, 100, 162}

\setbeamercolor{title separator}{bg=dirlightblue!50, fg=dirblue}

%Цвета блоков

% Голубой блок!
\setbeamercolor{block title}{bg=dirblue!30,fg=dirblack}
\setbeamercolor{block title example}{bg=dirlightblue!50,fg=dirblack}
\setbeamercolor{block body example}{bg=dirlightblue!20,fg=dirblack}

\AtBeginEnvironment{exampleblock}{\setbeamercolor{itemize item}{fg=dirblack}}
%\setbeamertemplate{blocks}[rounded][shadow]

% Набор команд для удобства верстки

% Набор команд для структуризации

%\newcommand{\quest}{\faQuestionCircleO}
%\faPencilSquareO \faPuzzlePiece \faQuestionCircleO  \faIcon*[regular]{file} {\textcolor{dirblue}
%\newcommand{\quest}{\textcolor{dirblue}{\boxed{\textbf{?}}}
%\newcommand{\task}{\faIcon{tasks}}
%\newcommand{\exmpl}{\faPuzzlePiece}
%\newcommand{\dfn}{\faIcon{pen-square}}
%\newcommand{\quest}{\textcolor{dirblue}{\faQuestionCircle[regular]}}
%\newcommand{\acc}[1]{\textcolor{dirred}{#1}}
%\newcommand{\accm}[1]{\textcolor{dirred}{#1}}
%\newcommand{\acct}[1]{\textcolor{dirblue}{#1}}
%\newcommand{\acctm}[1]{\textcolor{dirblue}{#1}}
%\newcommand{\accex}[1]{\textcolor{dirblack}{\bf #1}}
%\newcommand{\accexm}[1]{\textcolor{dirblack}{ \mathbf{#1}}}
%\newcommand{\acclp}[1]{\textcolor{dirorange}{\it #1}}
\newcommand{\todo}[1]{\textcolor{dircomment}{\bf #1}}
%\newcommand{\graylink}[1]{{\fontsize{11}{12}\selectfont \textcolor{gray}{#1}}}
%\newcommand{\figcaption}[1]{{\fontsize{18}{20}\selectfont #1}}


\newcommand{\videotitle}[1]{
    {\fontsize{33}{30}\selectfont \textcolor{dirblue}{\textbf{#1}} }

    %\todo{название видеофрагмента}
}

\newcommand{\lecturetitle}[1]{
  {\fontsize{33}{30}\selectfont \textcolor{dirblue}{\textbf{#1}} }

    %\todo{название лекции}
}





%\newcommand{\spcbig}{\vspace{-10 pt}}
%\newcommand{\spcsmall}{\vspace{-5 pt}}

%\usepackage{listings}
%\lstset{
%xleftmargin=0 pt,
%  basicstyle=\small, 
%  language=Python,
  %tabsize = 2,
%  backgroundcolor=\color{mc!20!white}
%}



%\newcommand{\mypart}[1]{\begin{frame}[standout]{\huge #1}\end{frame}}

\setbeamercolor{background canvas}{bg=}

% frame title setup
\setbeamercolor{frametitle}{bg=,fg=dirblue}
\setbeamertemplate{frametitle}[default][left]

\addtobeamertemplate{frametitle}{\hspace*{0.1 cm}}{\vspace*{0.25cm}}


%Шрифты
\setbeamerfont{frametitle}{family=\rmfamily,series=\bfseries,size={\fontsize{33}{30}}}
\setbeamerfont{framesubtitle}{family=\rmfamily,series=\bfseries,size={\fontsize{26}{20}}}


% удобнее знать номер слайда, чтобы вносить правки!  

\setbeamercolor{footline}{fg=dircomment}
\setbeamerfont{footline}{series=\bfseries, size={\fontsize{12}{14}}}
%\setbeamertemplate{footline}[page number]


\defbeamertemplate{footline}{custom footline}
{%
  \hspace*{\fill}%
  \usebeamercolor[fg]{page number in head/foot}%
  \usebeamerfont{page number in head/foot}%
  page: \insertpagenumber\,/\,\insertpresentationendpage%
  \hspace{20pt}%
  slide: \insertframenumber\,/\,\inserttotalframenumber%
  %\hspace*{\fill}
  \vskip2pt%
}
%\setbeamertemplate{footline}[custom footline]

\usepackage{physics}
\usepackage[makeroom]{cancel}



% tikz block

\usepackage{pgfplots}
\pgfplotsset{compat=newest}

\usepackage{tikz}
\usetikzlibrary{calc}
\usetikzlibrary{quotes,angles}
\usetikzlibrary{arrows}
\usetikzlibrary{arrows.meta}
\usetikzlibrary{positioning,intersections,decorations.markings}
\usetikzlibrary{patterns}

\usepackage{tkz-euclide} 
%\tikzset{>=latex}

\tikzset{cross/.style={cross out, draw=black, minimum size=2*(#1-\pgflinewidth), inner sep=0pt, outer sep=0pt},
%default radius will be 1pt. 
cross/.default={5pt}}

\colorlet{veca}{red}
\colorlet{vecb}{blue}
\colorlet{vecc}{olive}


\newcommand{\grid}{\draw[color=gray,step=1.0,dotted] (-2.1,-2.1) grid (9.6,6.1)}

% end tikz block

\newcommand{\R}{\mathbb{R}}
\newcommand{\Rot}{\mathrm{R}}
\newcommand{\HH}{\mathrm{H}}
\newcommand{\Id}{\mathrm{I}}
\newcommand{\RR}{\mathbb{R}}
\newcommand{\ZZ}{\mathbb{Z}}
\newcommand{\la}{\lambda}
\let\P\relax
\newcommand{\P}{\mathbb{P}}
\newcommand{\E}{\mathbb{E}}

\newcommand{\cN}{\mathcal{N}}
\newcommand{\dN}{\mathcal{N}}

\newcommand{\qL}{q_{\text{left}}}
\newcommand{\qR}{q_{\text{right}}}



\newcommand{\ba}{\mathbf{a}}
\newcommand{\be}{\mathbf{e}}
\newcommand{\bb}{\mathbf{b}}
\newcommand{\bc}{\mathbf{c}}
\newcommand{\bd}{\mathbf{d}}
\newcommand{\bx}{\mathbf{x}}
\newcommand{\bff}{\mathbf{f}} % \bf is already def
\newcommand{\bv}{\mathbf{v}}
\newcommand{\bzero}{\mathbf{0}}



\DeclareMathOperator{\Var}{Var}
\DeclareMathOperator{\sVar}{sVar}
\DeclareMathOperator{\Cov}{Cov}
\DeclareMathOperator{\sCov}{sCov}
\DeclareMathOperator{\sCorr}{sCorr}
\DeclareMathOperator{\pCorr}{pCorr}
\DeclareMathOperator{\Corr}{Corr}
\DeclareMathOperator{\Med}{Med}
\let\L\relax
\DeclareMathOperator{\L}{L}


\DeclareMathOperator{\plim}{plim}
\DeclareMathOperator{\sign}{sign}


\newcommand{\graylink}[1]{{\fontsize{11}{12}\selectfont \textcolor{gray}{#1}}}
\newcommand{\figcaption}[1]{{\fontsize{18}{20}\selectfont #1}}





\begin{document}


\begin{frame} % название лекции


    \lecturetitle{Forecasting}

\end{frame}

% !TEX root = ../om_ts_003.tex

\begin{frame} % frame name
	
	\videotitle{Forecasting without a model}
	
\end{frame}



\begin{frame}{Forecasting without a model: Plan}
	\begin{itemize}[<+->]
		\item Converting time series into \alert{cross-sectional} data
		\item Add \alert{lags} to the $y_t$ variable
		\item Use \alert{aggregation} and \alert{sliding} or \alert{growing} windows
	\end{itemize}
	
\end{frame}


\begin{frame}
	\frametitle{Forecasting}
	
	\begin{block}{Adding predictors}
		There are algorithms that, based on the training sample of the dependent variable $y$,
		learning matrix of predictors $X$, and new predictors $X_F$ build a forecast $\hat y_F$
	\end{block}
	
	\pause
	
	\alert{Random Forest}, \alert{gradient boosting}\ldots{ }\pause and even \alert{linear regression}!
	
	\pause
	
	You can \alert{average} ARIMA/ETS forecasts and forecasts from other algorithms
	
\end{frame}


\begin{frame}
	\frametitle{How to create predictors?}
	
	From one column $y$ you can create an entire matrix of  predictors~$X$!
	
	\begin{itemize}[<+->]
		\item Use \alert{lags} $y_{t-k}$
		\item Use \alert{functions of lags} as predictors
	\end{itemize}
	
	
\end{frame}

\begin{frame}
	\frametitle{Using $y$ lags}
	
	For example, let's take two lags, $Ly_t$ and $L^2 y_t$
	\pause
	
	\alert{Training} sample:
	\[
	\begin{pmatrix}
		y_3 \\
		y_4 \\
		y_5 \\
		\vdots \\
		y_T
	\end{pmatrix} \quad
	\begin{pmatrix}
		y_1 & y_2 \\
		y_2 & y_3 \\
		y_3 & y_4 \\
		\vdots & \vdots \\
		y_{T-2} & y_{T-1} \\
	\end{pmatrix}
	\]
	\pause
	Sample for \alert{prediction}:
	\[
	\begin{pmatrix}
		?
	\end{pmatrix} \quad
	\begin{pmatrix}
		y_{T-1} & y_{T} \\
	\end{pmatrix}
	\]
	
\end{frame}

\begin{frame}
	\frametitle{How many lags to add?}
	
	\begin{itemize}[<+->]
		\item Each added lag \alert{reduces} the training sample!
		\item It is reasonable to add \alert{closest lags} $Ly_t$, $L^2y_t$
		\item For seasonal data it is reasonable to add a \alert{seasonal lag} $L^{12} y_t$
		\item Some algorithms   are  \alert{sensitive to extra predictors} (e.g. regression)
		\item Some algorithms  are \alert{insensitive to extra predictors} (e.g. a random forest)
	\end{itemize}
\end{frame}

\begin{frame}
	\frametitle{Functions of lags}
	When predicting $y_{t}$ we can use any function of \alert{previous} $y_{t-1}$, $y_{t-2}$, \ldots lags
	
	\pause
	
	For example:
	\begin{itemize}[<+->]
		\item $\Delta y_{t-1} = y_{t-1} - y_{t-2}$;
		\item $\max\{ y_{t-1}, y_{t-2}, y_{t-3} \}$;
		\item $\min\{ y_{t-1}, y_{t-2}, \ldots, y_1\}$.
	\end{itemize}
\end{frame}

\begin{frame}
	\frametitle{Typical Predictor}
	
	\begin{itemize}[<+->]
		\item \alert{Aggregate function}:
		
		Min, Max, Mean, Median, Range, Sample Variance, Sample Standard Deviation, \ldots
		
		\item 		
		\alert{Sliding Window}: The aggregate function can be applied to, say, the previous three values $y_{t-1}$, $y_{t-2}$, $y_{t-3}$.
		
		\item 		
		\alert{Growing Window}: The aggregate function can be applied to all previous values $y_{t-1}$, $y_{t-2}$, \ldots, $y_{1}$.
	\end{itemize}
	
\end{frame}


\begin{frame}
	\frametitle{Using $y$ lag functions}
	
	For example, let's take the maximum as a sliding window and the minimum as a growing window
	\pause
	
	\alert{Training} sample:
	\[
	\begin{pmatrix}
		y_3 \\
		y_4 \\
		y_5 \\
		\vdots \\
		y_T
	\end{pmatrix} \quad
	\begin{pmatrix}
		\max\{y_1, y_2\} & \min\{y_1, y_2\} \\
		\max\{y_2, y_3\} & \min\{y_1, y_2, y_3\} \\
		\max\{y_3, y_4\} & \min\{y_1, \ldots, y_4\} \\
		\vdots & \vdots \\
		\max\{y_{T-2}, y_{T-1}\} & \min\{y_1, \ldots, y_{T-1}\} \\
	\end{pmatrix}
	\]
	\pause
	Sample for \alert{prediction}:
	\[
	\begin{pmatrix}
		?
	\end{pmatrix} \quad
	\begin{pmatrix}
		\max\{y_{T-1}, y_{T}\} & \min\{y_1, \ldots, y_{T}\} \\
	\end{pmatrix}
	\]
	
\end{frame}




\begin{frame}{Forecasting without a model: Summary}
	
	\begin{itemize}[<+->]
		\item Remember about \alert{random forest}, \alert{gradient boosting} and even \alert{regular regression}
		\item Add \alert{dependent variable lags}
		\item Add \alert{aggregation functions} as a sliding and growing window
	\end{itemize}
\end{frame}
% !TEX root = ../om_ts_003.tex

\begin{frame} % frame name
	
	\videotitle{More predictors!}
	
	% https://www.youtube.com/watch?v=7CCBsshm-cQ
	
\end{frame}



\begin{frame}{More predictors: Plan}
	\begin{itemize}[<+->]
		\item \alert{Trend} predictors
		\item \alert{Seasonal} and \alert{holiday} dummy
		\item \alert{Cosines} and \alert{sines} as predictors
	\end{itemize}
	
\end{frame}

\begin{frame}
	\frametitle{Let's use the time!}
	
	Let's take $t$ and $\sqrt{t}$ as an example
	\pause
	
	\alert{Training} sample:
	\[
	\begin{pmatrix}
		y_1 \\
		y_2 \\
		y_3 \\
		\vdots \\
		y_T
	\end{pmatrix} \quad
	\begin{pmatrix}
		1 & \sqrt{1} \\
		2 & \sqrt{2} \\
		3 & \sqrt{3} \\
		\vdots & \vdots \\
		T & \sqrt{T} \\
	\end{pmatrix}
	\]
	\pause
	Sample for \alert{prediction}:
	\[
	\begin{pmatrix}
		?
	\end{pmatrix} \quad
	\begin{pmatrix}
		T+1 & \sqrt{T+1} \\
	\end{pmatrix}
	\]
	
\end{frame}


\begin{frame}
	\frametitle{Monotonic transformations of time}
	
	\begin{itemize}[<+->]
		\item You can always \alert{try}!
		\item For algorithms based on  \alert{decision trees} (random forest, gradient boosting)
		additional monotonic time transformations are \alert{useless}
		\item Be aware of the possible transformation \alert{of the original variable} (logarithm, Box-Cox transformation)
	\end{itemize}
\end{frame}

\begin{frame}
	\frametitle{Seasonal and holiday dummy}
	
	If there are \alert{not many} seasons, then it is reasonable to include a dummy for each season
	
	\pause
	\alert{Training} sample for quarterly data:
	\[
	\begin{pmatrix}
		y_1 \\
		y_2 \\
		y_3 \\
		y_4 \\
		y_5 \\
		y_6 \\
		\vdots \\
		y_T
	\end{pmatrix} \quad
	\begin{pmatrix}
		1 & 0 & 0 & 0 \\
		0 & 1 & 0 & 0 \\
		0 & 0 & 1 & 0 \\
		0 & 0 & 0 & 1 \\
		1 & 0 & 0 & 0 \\
		0 & 1 & 0 & 0 \\
		\vdots & \vdots & \vdots & \vdots \\
		0 & 0 & 1 & 0 \\
	\end{pmatrix}
	\]
\end{frame}


\begin{frame}
	\frametitle{The dummy variable trap}
	
	
	In \alert{regression}, be aware of the dummy variable \alert{trap}! \pause
	
	\begin{itemize}[<+->]
		\item Either use a dummy for every season and a model without a constant,
		\item or use a dummy for all seasons except one and a model with a constant
	\end{itemize}
	
	\pause
	Algorithms based on  \alert{decision trees} (random forest, gradient boosting) are
	\alert{resistant} to the dummy variable trap
\end{frame}


\begin{frame}
	\frametitle{Why do we need sines and cosines?}
	
	Adding all dummy variables works \alert{poorly}  if you need \alert{a lot} of them. \pause
	
	It is hardly worth adding 365 dummy variables for \alert{daily} data. \pause
	
	Sine and cosine will help to decrease the  \alert{number} of predictors!
	
	\pause
	Two facts:
	\begin{itemize}[<+->]
		\item The period of $\sin t$ and $\cos t$ is $2\pi$
		\item Multiplying the argument by $a$ reduces the \alert{period} by a factor of $a$
	\end{itemize}
	
\end{frame}


\begin{frame}
	\frametitle{Fourier expansion}
	
	\begin{block}{Theorem}
		Any continuous and differentiable function $f$ with period $2\pi$ can be represented as
		\[
		f(t) = c + \sum_{k=1}^{\infty} a_k \cos(kt) + b_k \sin (kt)
		\]
	\end{block}
	
	\pause
	Practical recipe for daily data:
	\begin{itemize}[<+->]
		\item Add predictors $\cos\left(\frac{2\pi}{365} \cdot t\right)$ and $\sin\left(\frac{2\pi}{365} \cdot t\right) $
		\item Add predictors $\cos\left(\frac{2\pi}{365} \cdot 2t\right)$ and $\sin\left(\frac{2\pi}{365} \cdot 2t\right) $
		\item Add predictors $\cos\left(\frac{2\pi}{365} \cdot 3t\right)$ and $\sin\left(\frac{2\pi}{365} \cdot 3t\right) $
		\item \ldots
	\end{itemize}
	
	
	
	
\end{frame}

\begin{frame}{More predictors: Summary}
	
	\begin{itemize}[<+->]
		\item Use \alert{time} as a predictor
		\item Seasonality in predictors can be reflected using \alert{dummy variables} or
		using \alert{cosine} and \alert{sine} functions
	\end{itemize}
\end{frame}
% !TEX root = ../om_ts_003.tex

\begin{frame} % frame name
	
	\videotitle{Predictors and $ARIMA$}
	
\end{frame}

\begin{frame}{Predictors and $ARIMA$: Plan}
	\begin{itemize}[<+->]
		\item Regression with \alert{$ARMA$ errors}
		\item \alert{$ARMAX$ model}
		\item \alert{$ARDL$ model}
	\end{itemize}
\end{frame}

\begin{frame}
	\frametitle{Regression with $ARMA$ errors}
	
	\begin{block}{Equation}
		\[
		y_t = \beta_1 + \beta_2 a_t + \beta_3 b_t + \varepsilon_t,
		\]
		where $a_t$ and $b_t$ are \alert{predictors}
	\end{block}
	\pause
	\begin{itemize}[<+->]
		\item The series $(y_t)$, $(a_t)$, $(b_t)$, $(\varepsilon_t)$ \alert{stationary}
		\item $\varepsilon_t \sim ARMA(p, q)$ with respect to white noise $(u_t)$
		\item $\E(\varepsilon_t \mid a_{t-1}, b_{t-1}, a_{t-2}, b_{t-2}, \ldots) = 0$
		\item \alert{Fourth moments} of predictors are finite
	\end{itemize}
	
\end{frame}


\begin{frame}
	\frametitle{$ARMAX$ model}
	
	\begin{block}{Equation}
		\[
		y_t = c + \gamma_1 a_t + \gamma_2 b_t + \beta_1 y_{t-1} + \ldots + \beta_p y_{t-p} + 
		\]
				\[
		 + u_t + \alpha_1 u_{t-1} + \ldots + \alpha_q u_{t-q },
		\]
		
		where $a_t$ and $b_t$ are \alert{predictors} and $(u_t)$ is \alert{white noise}
	\end{block}
	\pause
	\begin{itemize}[<+->]
		\item Series $(y_t)$, $(a_t)$, $(b_t)$ \alert{stationary}
		\item $\E(u_t \mid a_{t-1}, b_{t-1}, y_{t-1}, a_{t-2}, b_{t-2}, y_{t-2} , \ldots) = 0$
		\item \alert{The fourth predictor moments} are finite
	\end{itemize}
	\pause
	$ARMAX$ model is not completely equivalent to regression with $ARMA$ errors, but gives \alert{approximately the same} quality of predictions
	
\end{frame}


\begin{frame}
	\frametitle{Properties of the $ARMAX$ model and regression with $ARMA$ errors}
	
	\begin{itemize}[<+->]
		\item If the model assumptions aren't violated, then the maximum likelihood estimators \alert{are consistent}
		\item For non-stationary variables $(y_t)$ and predictors $(a_t)$ and $(b_t)$, we can switch to the first \alert{differences}
		\item Estimators \alert{remain} consistent if trend, seasonality dummy and trigonometric predictors are added
		\item \alert{Not any} predictor makes it possible to obtain a consistent estimator of the coefficient
		\item \alert{Sometimes} you can get good predictions even if the assumptions are violated
	\end{itemize}
	
\end{frame}



\begin{frame}
	\frametitle{$ARDL$ model}
	
	\alert{ARDL} — \alert{A}uto\alert{R}egressive \alert{D}istributed \alert{L}ag model
	
	Autoregressive model with distributed lags
	
	\begin{block}{The $ARDL(p, q)$ model equation}
		\[
		y_t = c + \beta_1 y_{t-1} + \ldots + \beta_p y_{t-p} + 
		\]
		\[
		+ x_t + \alpha_1 x_{t-1} + \ldots + \alpha_q x_{t-q} + u_t
		\]

	\end{block}
	\pause
	\begin{itemize}[<+->]
		\item $(u_t)$ errors are \alert{white noise}
		\item Process $(x_t)$ \alert{or} process $(\Delta x_t)$ is stationary
		\item Process $(y_t)$ is \alert{non-stationary}, but $(\Delta y_t)$ is stationary
		\item $\E(u_t \mid y_{t-1}, x_{t-1}, y_{t-2}, x_{t-2}, \ldots) = 0$
	\end{itemize}
	
\end{frame}

\begin{frame}
	\frametitle{Properties of the $ARDL$ model}
	\begin{itemize}[<+->]
		\item Using \alert{predictor lags} $(x_t)$ instead of noise lags $(u_t)$ 
		\item Suitable for \alert{non-stationary} $(y_t)$
		\item Used to find a \alert{long-term relationships} between time series
		\item If  the model assumptions aren't violated, then \alert{the OLS estimators are consistent}, although they are biased
		\item You can add \alert{multiple predictors} with different number of lags
	\end{itemize}
	
\end{frame}


\begin{frame}{Predictors and $ARIMA$: Summary}
	\begin{itemize}[<+->]
		\item For \alert{stationary data} you can use regression with $ARMA$ errors or $ARMAX$ model
		\item Regression with $ARMA$ errors can be constructed for the   \alert{differences}
		\item For \alert{non-stationary series} it is sometimes possible to use the $ARDL$ model
	\end{itemize}
	
\end{frame}
% !TEX root = ../om_ts_003.tex

\begin{frame} % frame name
	
	\videotitle{Model Quality}
	
\end{frame}



\begin{frame}{Model Quality: Plan}
	\begin{itemize}[<+->]
		\item Scale-based metrics
		\item Percentage-based metrics
%		\item Akaike Criterion.
	\end{itemize}
	
\end{frame}


\begin{frame}
	\frametitle{Remember the goal!}
	
	
	
	If the goal of building a model is forecasts one step ahead,
	then it is reasonable to compare models in predictive strength one step ahead.
	
	\pause
	\
	
	If the goal is to detect  the moment of model discord,
	then it is reasonable to look for a model that gives the minimum error when there is no discord,
	and the maximum error when there is discord.
	
\end{frame}

\begin{frame}
	\frametitle{Notations for brevity}
	
	For the forecast, it is important \alert{when} it is built, and for \alert{how many steps ahead}:
	\[
	\hat y_{t+h \mid t}
	\]
	
	\pause
	Sometimes for \alert{short}:
	\[
	\hat y_{t+h}
	\]
	\pause
	Problem:
	\[
	\hat y_{(t+1) + 2} \neq \hat y_{(t+2) + 1}
	\]
	
\end{frame}


\begin{frame}
	\frametitle{Anti-quality metrics}
	
	\alert{Forecast error}: $e_{t+h} = y_{t+h} - \hat y_{t+h}$
	
	\pause
	\alert{Mean Absolute Error}:
	\[
	MAE = \frac{\abs {e_{T+1}} + \abs{e_{T+2}}+ \ldots + \abs{e_{T+H}} }{H}
	\]
	\pause
	\alert{Root Mean Squared Error}:
	\[
	RMSE = \sqrt{ \frac{e^2_{T+1} + e^2_{T+2}+ \ldots + e^2_{T+H} }{H} }
	\]
	
\end{frame}


\begin{frame}
	\frametitle{Scaling}
	
	Convert error $e_{t+h} = y_{t+h} - \hat y_{t+h}$ \alert{to percentage} $p_t= e_t/y_t \cdot 100$ or
	$p^s_t = e_t/(0.5 y_t + 0.5\hat y_t) \cdot 100$
	
	\pause
	\alert{Mean Absolute Percentage Error}:
	\[
	MAPE = \frac{\abs {p_{T+1}} + \abs{p_{T+2}}+ \ldots + \abs{p_{T+H}} }{H}
	\]
	\pause
	\alert{Symmetric Mean Absolute Percentage Error}:
	\[
	sMAPE = \frac{\abs {p^s_{T+1}} + \abs{p^s_{T+2}}+ \ldots + \abs{p^s_{T+H}} }{H} 
	\]
	
\end{frame}

\begin{frame}
	\frametitle{Automatically compare with naive model}
	
	\alert{Naive}: $\hat y^{naive}_t = y_{t-1}$ or $\hat y^{naive}_t = y_{t-12}$
	\pause
	Let's scale our forecast error $e_t$ to $MAE^{naive}$:
	\[
	q_t = \frac{e_t}{MAE^{naive}}
	\]
	
	\pause
	\alert{Mean Absolute Scaled Error}:
	\[
	MASE = \frac{\abs {q_{T+1}} + \abs{q_{T+2}}+ \ldots + \abs{q_{T+H}} }{H}
	\]
	
	\pause
	Comparing $q$ to 1 compares our model with the naive one
	
	
\end{frame}



\begin{frame}{Model Quality: Summary}
	
	\begin{itemize}[<+->]
		\item MAE, RMSE
		\item MAPE
		\item MASE
	\end{itemize}
\end{frame}




\begin{frame} % frame name
	
	\videotitle{Model Comparison}
	
\end{frame}



\begin{frame}{Model Comparison: Plan}
	\begin{itemize}[<+->]
		\item Cross-validation
		\item Akaike criterion
	\end{itemize}
	
\end{frame}




\begin{frame}
	\frametitle{Train and test set}
	
	Strategy:
	\begin{enumerate}[<+->]
		\item Split the entire sample into \alert{training} (at the beginning) and \alert{test} (at the end) sets
		\item \alert{Evaluate} several models on the training set
		\item \alert{Predict} each observation of the test sample using each model
		\item Calculate prediction \alert{errors}
		\item \alert{Compare} models by $MAE$ and choose the best one
	\end{enumerate}
	
	\pause
	Disadvantage: \alert{forecasts have different horizons}
	
\end{frame}

\begin{frame}
	\frametitle{Dividing into train and test}
	    
        \begin{tikzpicture}[
            roundnode/.style={circle, draw=black!60, fill=black!20, very thick, minimum size=7mm},
            rednode/.style={circle, draw=orange!60, fill=orange!20, very thick, minimum size=7mm},
            unusednode/.style={circle, draw=black!60, fill=black!2, very thick, minimum size=5mm},
            ]
            \node[roundnode] (1) {};
            \node[roundnode] (2) [right=of 1] {};
            \node[roundnode] (3) [right=of 2] {};
            \node[roundnode] (4) [right=of 3] {};
            \node[roundnode] (5) [right=of 4] {};
            \node[rednode] (6) [right=of 5] {};
            \node[unusednode] (7) [right=of 6] {};
            \node[unusednode] (8) [right=of 7] {};
            \node[unusednode] (9) [right=of 8] {};
            \node[unusednode] (10) [right=of 9] {};
            \draw[->] (5.north) to [out=30,in=150] (6.north);
            \draw (1) -- (2);
            \draw (2) -- (3);
            \draw (3) -- (4);
            \draw (4) -- (5);
    
        \end{tikzpicture}
            
        \begin{tikzpicture}[
            roundnode/.style={circle, draw=black!60, fill=black!20, very thick, minimum size=7mm},
            rednode/.style={circle, draw=orange!60, fill=orange!20, very thick, minimum size=7mm},
            unusednode/.style={circle, draw=black!60, fill=black!2, very thick, minimum size=5mm},
            ]
            \node[roundnode] (1) {};
            \node[roundnode] (2) [right=of 1] {};
            \node[roundnode] (3) [right=of 2] {};
            \node[roundnode] (4) [right=of 3] {};
            \node[roundnode] (5) [right=of 4] {};
            \node[unusednode] (6) [right=of 5] {};
            \node[rednode] (7) [right=of 6] {};
            \node[unusednode] (8) [right=of 7] {};
            \node[unusednode] (9) [right=of 8] {};
            \node[unusednode] (10) [right=of 9] {};
            \draw[->] (5.north) to [out=30,in=150] (7.north);
            \draw (1) -- (2);
            \draw (2) -- (3);
            \draw (3) -- (4);
            \draw (4) -- (5);
    
        \end{tikzpicture}
    
        \begin{tikzpicture}[
            roundnode/.style={circle, draw=black!60, fill=black!20, very thick, minimum size=7mm},
            rednode/.style={circle, draw=orange!60, fill=orange!20, very thick, minimum size=7mm},
            unusednode/.style={circle, draw=black!60, fill=black!2, very thick, minimum size=5mm},
            ]
            \node[roundnode] (1) {};
            \node[roundnode] (2) [right=of 1] {};
            \node[roundnode] (3) [right=of 2] {};
            \node[roundnode] (4) [right=of 3] {};
            \node[roundnode] (5) [right=of 4] {};
            \node[unusednode] (6) [right=of 5] {};
            \node[unusednode] (7) [right=of 6] {};
            \node[rednode] (8) [right=of 7] {};
            \node[unusednode] (9) [right=of 8] {};
            \node[unusednode] (10) [right=of 9] {};
            \draw[->] (5.north) to [out=30,in=150] (8.north);
            \draw (1) -- (2);
            \draw (2) -- (3);
            \draw (3) -- (4);
            \draw (4) -- (5);
        \end{tikzpicture}
    
        \begin{tikzpicture}[
            roundnode/.style={circle, draw=black!60, fill=black!20, very thick, minimum size=7mm},
            rednode/.style={circle, draw=orange!60, fill=orange!20, very thick, minimum size=7mm},
            unusednode/.style={circle, draw=black!60, fill=black!2, very thick, minimum size=5mm},
            ]
            \node[roundnode] (1) {};
            \node[roundnode] (2) [right=of 1] {};
            \node[roundnode] (3) [right=of 2] {};
            \node[roundnode] (4) [right=of 3] {};
            \node[roundnode] (5) [right=of 4] {};
            \node[unusednode] (6) [right=of 5] {};
            \node[unusednode] (7) [right=of 6] {};
            \node[unusednode] (8) [right=of 7] {};
            \node[rednode] (9) [right=of 8] {};
            \node[unusednode] (10) [right=of 9] {};
            \draw[->] (5.north) to [out=30,in=150] (9.north);
            \draw (1) -- (2);
            \draw (2) -- (3);
            \draw (3) -- (4);
            \draw (4) -- (5);
        \end{tikzpicture}
    
        \begin{tikzpicture}[
            roundnode/.style={circle, draw=black!60, fill=black!20, very thick, minimum size=7mm},
            rednode/.style={circle, draw=orange!60, fill=orange!20, very thick, minimum size=7mm},
            unusednode/.style={circle, draw=black!60, fill=black!2, very thick, minimum size=5mm},
            ]
            \node[roundnode] (1) {};
            \node[roundnode] (2) [right=of 1] {};
            \node[roundnode] (3) [right=of 2] {};
            \node[roundnode] (4) [right=of 3] {};
            \node[roundnode] (5) [right=of 4] {};
            \node[unusednode] (6) [right=of 5] {};
            \node[unusednode] (7) [right=of 6] {};
            \node[unusednode] (8) [right=of 7] {};
            \node[unusednode] (9) [right=of 8] {};
            \node[rednode] (10) [right=of 9] {};
            \draw[->] (5.north) to [out=30,in=150] (10.north);
            \draw (1) -- (2);
            \draw (2) -- (3);
            \draw (3) -- (4);
            \draw (4) -- (5);
        \end{tikzpicture}
    
    
\end{frame}

\begin{frame}
	\frametitle{Sliding window Cross-validation}
	
	Strategy:
	\begin{enumerate}[<+->]
		\item Select the starting size for  \alert{train} sample (at the beginning)
		\item \alert{Evaluate} several models on the train set
		\item \alert{Predict} one step ahead with each model
		\item Calculate  prediction \alert{errors}
		\item \alert{Shift} the training sample one observation to the right
		\item Repeat steps 2-5
		\item \alert{Compare} models by $MAE$ and choose the best one
	\end{enumerate}
	
\end{frame}


\begin{frame}
	\frametitle{Sliding window Cross-validation}
	
    \begin{tikzpicture}[
        roundnode/.style={circle, draw=black!60, fill=black!20, very thick, minimum size=7mm},
        rednode/.style={circle, draw=orange!60, fill=orange!20, very thick, minimum size=7mm},
        unusednode/.style={circle, draw=black!60, fill=black!2, very thick, minimum size=5mm},
        ]
        \node[roundnode] (1) {};
        \node[roundnode] (2) [right=of 1] {};
        \node[roundnode] (3) [right=of 2] {};
        \node[roundnode] (4) [right=of 3] {};
        \node[roundnode] (5) [right=of 4] {};
        \node[rednode] (6) [right=of 5] {};
        \node[unusednode] (7) [right=of 6] {};
        \node[unusednode] (8) [right=of 7] {};
        \node[unusednode] (9) [right=of 8] {};
        \node[unusednode] (10) [right=of 9] {};
        \draw[->] (5.north) to [out=30,in=150] (6.north);
        \draw (1) -- (2);
        \draw (2) -- (3);
        \draw (3) -- (4);
        \draw (4) -- (5);

    \end{tikzpicture}
        
    \begin{tikzpicture}[
        roundnode/.style={circle, draw=black!60, fill=black!20, very thick, minimum size=7mm},
        rednode/.style={circle, draw=orange!60, fill=orange!20, very thick, minimum size=7mm},
        unusednode/.style={circle, draw=black!60, fill=black!2, very thick, minimum size=5mm},
        ]
        \node[unusednode] (1) {};
        \node[roundnode] (2) [right=of 1] {};
        \node[roundnode] (3) [right=of 2] {};
        \node[roundnode] (4) [right=of 3] {};
        \node[roundnode] (5) [right=of 4] {};
        \node[roundnode] (6) [right=of 5] {};
        \node[rednode] (7) [right=of 6] {};
        \node[unusednode] (8) [right=of 7] {};
        \node[unusednode] (9) [right=of 8] {};
        \node[unusednode] (10) [right=of 9] {};
        \draw[->] (6.north) to [out=30,in=150] (7.north);
        \draw (2) -- (3);
        \draw (3) -- (4);
        \draw (4) -- (5);
        \draw (5) -- (6);

    \end{tikzpicture}

    \begin{tikzpicture}[
        roundnode/.style={circle, draw=black!60, fill=black!20, very thick, minimum size=7mm},
        rednode/.style={circle, draw=orange!60, fill=orange!20, very thick, minimum size=7mm},
        unusednode/.style={circle, draw=black!60, fill=black!2, very thick, minimum size=5mm},
        ]
        \node[unusednode] (1) {};
        \node[unusednode] (2) [right=of 1] {};
        \node[roundnode] (3) [right=of 2] {};
        \node[roundnode] (4) [right=of 3] {};
        \node[roundnode] (5) [right=of 4] {};
        \node[roundnode] (6) [right=of 5] {};
        \node[roundnode] (7) [right=of 6] {};
        \node[rednode] (8) [right=of 7] {};
        \node[unusednode] (9) [right=of 8] {};
        \node[unusednode] (10) [right=of 9] {};
        \draw[->] (7.north) to [out=30,in=150] (8.north);
        \draw (3) -- (4);
        \draw (4) -- (5);
        \draw (5) -- (6);
        \draw (6) -- (7);
    \end{tikzpicture}

    \begin{tikzpicture}[
        roundnode/.style={circle, draw=black!60, fill=black!20, very thick, minimum size=7mm},
        rednode/.style={circle, draw=orange!60, fill=orange!20, very thick, minimum size=7mm},
        unusednode/.style={circle, draw=black!60, fill=black!2, very thick, minimum size=5mm},
        ]
        \node[unusednode] (1) {};
        \node[unusednode] (2) [right=of 1] {};
        \node[unusednode] (3) [right=of 2] {};
        \node[roundnode] (4) [right=of 3] {};
        \node[roundnode] (5) [right=of 4] {};
        \node[roundnode] (6) [right=of 5] {};
        \node[roundnode] (7) [right=of 6] {};
        \node[roundnode] (8) [right=of 7] {};
        \node[rednode] (9) [right=of 8] {};
        \node[unusednode] (10) [right=of 9] {};
        \draw[->] (8.north) to [out=30,in=150] (9.north);
        \draw (4) -- (5);
        \draw (5) -- (6);
        \draw (6) -- (7);
        \draw (7) -- (8);
    \end{tikzpicture}

    \begin{tikzpicture}[
        roundnode/.style={circle, draw=black!60, fill=black!20, very thick, minimum size=7mm},
        rednode/.style={circle, draw=orange!60, fill=orange!20, very thick, minimum size=7mm},
        unusednode/.style={circle, draw=black!60, fill=black!2, very thick, minimum size=5mm},
        ]
        \node[unusednode] (1) {};
        \node[unusednode] (2) [right=of 1] {};
        \node[unusednode] (3) [right=of 2] {};
        \node[unusednode] (4) [right=of 3] {};
        \node[roundnode] (5) [right=of 4] {};
        \node[roundnode] (6) [right=of 5] {};
        \node[roundnode] (7) [right=of 6] {};
        \node[roundnode] (8) [right=of 7] {};
        \node[roundnode] (9) [right=of 8] {};
        \node[rednode] (10) [right=of 9] {};
        \draw[->] (9.north) to [out=30,in=150] (10.north);
        \draw (5) -- (6);
        \draw (6) -- (7);
        \draw (7) -- (8);
        \draw (8) -- (9);
    \end{tikzpicture}


\end{frame}

\begin{frame}
	\frametitle{Growing window Cross-validation}
	
	Strategy:
	\begin{enumerate}
		\item Select the starting size for  \alert{train} sample (at the beginning)
		\item Evaluate several models on the training set
		\item Predict one step ahead with each model
		\item Calculate prediction errors
		\item \alert{Increase} the training set by one observation.
		\item Repeat steps 2-5
		\item Compare models by $MAE$ and choose the best one
	\end{enumerate}
	
\end{frame}


\begin{frame}
	\frametitle{Growing window Cross-validation}

    \begin{tikzpicture}[
        roundnode/.style={circle, draw=black!60, fill=black!20, very thick, minimum size=7mm},
        rednode/.style={circle, draw=orange!60, fill=orange!20, very thick, minimum size=7mm},
        unusednode/.style={circle, draw=black!60, fill=black!2, very thick, minimum size=5mm},
        ]
        \node[roundnode] (1) {};
        \node[roundnode] (2) [right=of 1] {};
        \node[roundnode] (3) [right=of 2] {};
        \node[roundnode] (4) [right=of 3] {};
        \node[roundnode] (5) [right=of 4] {};
        \node[rednode] (6) [right=of 5] {};
        \node[unusednode] (7) [right=of 6] {};
        \node[unusednode] (8) [right=of 7] {};
        \node[unusednode] (9) [right=of 8] {};
        \node[unusednode] (10) [right=of 9] {};
        \draw[->] (5.north) to [out=30,in=150] (6.north);
        \draw (1) -- (2);
        \draw (2) -- (3);
        \draw (3) -- (4);
        \draw (4) -- (5);

    \end{tikzpicture}
        
    \begin{tikzpicture}[
        roundnode/.style={circle, draw=black!60, fill=black!20, very thick, minimum size=7mm},
        rednode/.style={circle, draw=orange!60, fill=orange!20, very thick, minimum size=7mm},
        unusednode/.style={circle, draw=black!60, fill=black!2, very thick, minimum size=5mm},
        ]
        \node[roundnode] (1) {};
        \node[roundnode] (2) [right=of 1] {};
        \node[roundnode] (3) [right=of 2] {};
        \node[roundnode] (4) [right=of 3] {};
        \node[roundnode] (5) [right=of 4] {};
        \node[roundnode] (6) [right=of 5] {};
        \node[rednode] (7) [right=of 6] {};
        \node[unusednode] (8) [right=of 7] {};
        \node[unusednode] (9) [right=of 8] {};
        \node[unusednode] (10) [right=of 9] {};
        \draw[->] (6.north) to [out=30,in=150] (7.north);
        \draw (1) -- (2);
        \draw (2) -- (3);
        \draw (3) -- (4);
        \draw (4) -- (5);
        \draw (5) -- (6);

    \end{tikzpicture}

    \begin{tikzpicture}[
        roundnode/.style={circle, draw=black!60, fill=black!20, very thick, minimum size=7mm},
        rednode/.style={circle, draw=orange!60, fill=orange!20, very thick, minimum size=7mm},
        unusednode/.style={circle, draw=black!60, fill=black!2, very thick, minimum size=5mm},
        ]
        \node[roundnode] (1) {};
        \node[roundnode] (2) [right=of 1] {};
        \node[roundnode] (3) [right=of 2] {};
        \node[roundnode] (4) [right=of 3] {};
        \node[roundnode] (5) [right=of 4] {};
        \node[roundnode] (6) [right=of 5] {};
        \node[roundnode] (7) [right=of 6] {};
        \node[rednode] (8) [right=of 7] {};
        \node[unusednode] (9) [right=of 8] {};
        \node[unusednode] (10) [right=of 9] {};
        \draw[->] (7.north) to [out=30,in=150] (8.north);
        \draw (1) -- (2);
        \draw (2) -- (3);
        \draw (3) -- (4);
        \draw (4) -- (5);
        \draw (5) -- (6);
        \draw (6) -- (7);
    \end{tikzpicture}

    \begin{tikzpicture}[
        roundnode/.style={circle, draw=black!60, fill=black!20, very thick, minimum size=7mm},
        rednode/.style={circle, draw=orange!60, fill=orange!20, very thick, minimum size=7mm},
        unusednode/.style={circle, draw=black!60, fill=black!2, very thick, minimum size=5mm},
        ]
        \node[roundnode] (1) {};
        \node[roundnode] (2) [right=of 1] {};
        \node[roundnode] (3) [right=of 2] {};
        \node[roundnode] (4) [right=of 3] {};
        \node[roundnode] (5) [right=of 4] {};
        \node[roundnode] (6) [right=of 5] {};
        \node[roundnode] (7) [right=of 6] {};
        \node[roundnode] (8) [right=of 7] {};
        \node[rednode] (9) [right=of 8] {};
        \node[unusednode] (10) [right=of 9] {};
        \draw[->] (8.north) to [out=30,in=150] (9.north);
        \draw (1) -- (2);
        \draw (2) -- (3);
        \draw (3) -- (4);
        \draw (4) -- (5);
        \draw (5) -- (6);
        \draw (6) -- (7);
        \draw (7) -- (8);
    \end{tikzpicture}

    \begin{tikzpicture}[
        roundnode/.style={circle, draw=black!60, fill=black!20, very thick, minimum size=7mm},
        rednode/.style={circle, draw=orange!60, fill=orange!20, very thick, minimum size=7mm},
        unusednode/.style={circle, draw=black!60, fill=black!2, very thick, minimum size=5mm},
        ]
        \node[roundnode] (1) {};
        \node[roundnode] (2) [right=of 1] {};
        \node[roundnode] (3) [right=of 2] {};
        \node[roundnode] (4) [right=of 3] {};
        \node[roundnode] (5) [right=of 4] {};
        \node[roundnode] (6) [right=of 5] {};
        \node[roundnode] (7) [right=of 6] {};
        \node[roundnode] (8) [right=of 7] {};
        \node[roundnode] (9) [right=of 8] {};
        \node[rednode] (10) [right=of 9] {};
        \draw[->] (9.north) to [out=30,in=150] (10.north);
        \draw (1) -- (2);
        \draw (2) -- (3);
        \draw (3) -- (4);
        \draw (4) -- (5);
        \draw (5) -- (6);
        \draw (6) -- (7);
        \draw (7) -- (8);
        \draw (8) -- (9);
    \end{tikzpicture}


\end{frame}

\begin{frame}
	\frametitle{Cross-validation Discussion}
	
	\alert{Sliding} window cross-validation:  there are many observations and we suspect that dependencies between values can change.
	
	\pause
	\
	
	\alert{Growing} window cross-validation: there are few observations or we are sure that the dependency persists.
	
	\pause
	\
		
	Cross-validation can be \alert{time consuming}!
	
\end{frame}

\begin{frame}
	\frametitle{Let's make cross-validation quicker!}
	
	Approximate cross-validation by one step forward based on  $RMSE$ using...
	\alert{Akaike Information Criterion}:
	\pause
	\[
	AIC = -2\ln L + 2k,
	\]
	where $\ln L$ is the logarithm of the maximum likelihood on the training set, $k$ is the total number of model parameters
	
\end{frame}

\begin{frame}
	\frametitle{Nuances of $AIC$}
	
	\begin{itemize}[<+->]
		\item $AIC$ has \alert{theoretical grounds}:
		\[
		\frac{AIC_A - AIC_B}{2} \approx KL(\text{Truth} || \text{Model A}) - \]
		\[
		- KL(\text{Truth} || \text{Model B})
		\]
		\item May be used \alert{for non-nested models}
		\item For Gaussian $y_t$ models, the criterion approximates \alert{comparison over $RMSE$}
		\item The models being compared must be for the \alert{same} observations
	\end{itemize}
	
	
	
\end{frame}


\begin{frame}{Model Comparison: Summary}
	
	\begin{itemize}[<+->]
		\item Cross-validation: sliding and growing window
		\item AIC is a fast and approximate analogue of cross-validation
	\end{itemize}
\end{frame}

% !TEX root = ../om_ts_003.tex

\begin{frame} % frame name
	
	\videotitle{Forecast comparison}
	
\end{frame}



\begin{frame}{Forecast comparison: Plan}
	\begin{itemize}[<+->]
		\item Diebold-Mariano test
		\item Test \alert{assumptions}
		\item Test implementation
	\end{itemize}
	
\end{frame}

\begin{frame}
	\frametitle{Diebold-Mariano Test}
	
	\begin{itemize}[<+->]
		\item Used to compare \alert{two} forecasts
		\item Compares the forecasts for the specified \alert{forecast  horizon} $h$ 
		\item Not optimal for \alert{model comparison}
		\item Not suitable for \alert{pairwise} comparison of multiple forecasts
	\end{itemize}
	
\end{frame}

\begin{frame}
	\frametitle{DM test assumptions}
	
	Consider difference of two \alert{ forecast losses}:
	\[
	d_t = e_{A,t}^2 - e_{B,t}^2, \quad e_{\text{Model},t} = \hat y_{\text{Model},t} - y_t
	\]
	\pause
	Difference $d_t$ assumed to be \alert{stationary}:\pause
	\[
	\E(d_t) = \mu_d,
	\]
	\pause
	\[
	\Cov(d_t, d_{t-k}) = \gamma_k,
	\]\pause
	in particular,
	\[
	\Var(d_t) = \gamma_0
	\]
	
\end{frame}

\begin{frame}
	\frametitle{Test method}
	Under the correct $H_0: \mu_d = 0$:
	\[
	DM = \frac{\bar d}{se(\bar d)} \to \cN(0;1),
	\]
	where $se^2(\bar d)$ is a consistent estimator for $\Var(\bar d)$
	\pause
	
	In practice, we  evaluate the regression on a constant
	\[
	\hat d_t = \hat \beta_1,
	\]
	\pause
	get $\hat\beta_1 = \bar d$ and use  \alert{robust standard errors},
	\[
	DM = \frac{\hat \beta_1}{se_{HAC}(\hat\beta_1)}.
	\]
\end{frame}

\begin{frame}
	\frametitle{How does robust estimators work?}
	
	Compare forecasts for $P$ steps ahead,
	\[
	\Var(\bar d) = \frac{\Var(d_1) + \Var(d_2) + \ldots + 2\Cov(d_1, d_2) + \ldots}{P^2}
	\]\pause
	From the stationarity of $d_t$, the variance of $\Var(\bar d)$ is
	\[
	\frac{P\gamma_0 + 2(P-1) \gamma_1 + 2(P-2)\gamma_2 +\ldots}{P^2}
	\]\pause
	The naive estimator for the variance of $\widehat\Var(\bar d)$ is
	\[
	\frac{ P\hat\gamma_0 + 2(P-1) \hat\gamma_1 + 2(P-2)\hat\gamma_2 +\ldots}{P^2}
	\]
\end{frame}



\begin{frame}
	\frametitle{Why compare forecasts?}
	
	\alert{Nuance}: comparison of forecasts and comparison of models are different tasks.
	\pause
	
	\
	
	A model can win a lot \alert{in simplicity} and lose a little in predictions.
	\pause
	
	\
	
	On a small sample \alert{loss of information} about the quality of forecasts on the training sample is significant.
	\pause
	

\end{frame}


\begin{frame}{Forecast comparison: Summary}
	
	\begin{itemize}[<+->]
		\item The Diebold-Mariano test is suitable for comparing \alert{two} forecasts
		\item Comparing forecasts and comparing models are \alert{slightly different} tasks
	\end{itemize}
\end{frame}



\end{document}
