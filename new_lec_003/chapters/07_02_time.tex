% !TEX root = ../om_ts_003.tex

\begin{frame} % frame name
	
	\videotitle{More predictors!}
	
	% https://www.youtube.com/watch?v=7CCBsshm-cQ
	
\end{frame}



\begin{frame}{More predictors: Plan}
	\begin{itemize}[<+->]
		\item \alert{Trend} predictors
		\item \alert{Seasonal} and \alert{holiday} dummy
		\item \alert{Cosines} and \alert{sines} as predictors
	\end{itemize}
	
\end{frame}

\begin{frame}
	\frametitle{Let's use the time!}
	
	Let's take $t$ and $\sqrt{t}$ as an example
	\pause
	
	\alert{Training} sample:
	\[
	\begin{pmatrix}
		y_1 \\
		y_2 \\
		y_3 \\
		\vdots \\
		y_T
	\end{pmatrix} \quad
	\begin{pmatrix}
		1 & \sqrt{1} \\
		2 & \sqrt{2} \\
		3 & \sqrt{3} \\
		\vdots & \vdots \\
		T & \sqrt{T} \\
	\end{pmatrix}
	\]
	\pause
	Sample for \alert{prediction}:
	\[
	\begin{pmatrix}
		?
	\end{pmatrix} \quad
	\begin{pmatrix}
		T+1 & \sqrt{T+1} \\
	\end{pmatrix}
	\]
	
\end{frame}


\begin{frame}
	\frametitle{Monotonic transformations of time}
	
	\begin{itemize}[<+->]
		\item You can always \alert{try}!
		\item For algorithms based on  \alert{decision trees} (random forest, gradient boosting)
		additional monotonic time transformations are \alert{useless}
		\item Be aware of the possible transformation \alert{of the original variable} (logarithm, Box-Cox transformation)
	\end{itemize}
\end{frame}

\begin{frame}
	\frametitle{Seasonal and holiday dummy}
	
	If there are \alert{not many} seasons, then it is reasonable to include a dummy for each season
	
	\pause
	\alert{Training} sample for quarterly data:
	\[
	\begin{pmatrix}
		y_1 \\
		y_2 \\
		y_3 \\
		y_4 \\
		y_5 \\
		y_6 \\
		\vdots \\
		y_T
	\end{pmatrix} \quad
	\begin{pmatrix}
		1 & 0 & 0 & 0 \\
		0 & 1 & 0 & 0 \\
		0 & 0 & 1 & 0 \\
		0 & 0 & 0 & 1 \\
		1 & 0 & 0 & 0 \\
		0 & 1 & 0 & 0 \\
		\vdots & \vdots & \vdots & \vdots \\
		0 & 0 & 1 & 0 \\
	\end{pmatrix}
	\]
\end{frame}


\begin{frame}
	\frametitle{The dummy variable trap}
	
	
	In \alert{regression}, be aware of the dummy variable \alert{trap}! \pause
	
	\begin{itemize}[<+->]
		\item Either use a dummy for every season and a model without a constant,
		\item or use a dummy for all seasons except one and a model with a constant
	\end{itemize}
	
	\pause
	Algorithms based on  \alert{decision trees} (random forest, gradient boosting) are
	\alert{resistant} to the dummy variable trap
\end{frame}


\begin{frame}
	\frametitle{Why do we need sines and cosines?}
	
	Adding all dummy variables works \alert{poorly}  if you need \alert{a lot} of them. \pause
	
	It is hardly worth adding 365 dummy variables for \alert{daily} data. \pause
	
	Sine and cosine will help to decrease the  \alert{number} of predictors!
	
	\pause
	Two facts:
	\begin{itemize}[<+->]
		\item The period of $\sin t$ and $\cos t$ is $2\pi$
		\item Multiplying the argument by $a$ reduces the \alert{period} by a factor of $a$
	\end{itemize}
	
\end{frame}


\begin{frame}
	\frametitle{Fourier expansion}
	
	\begin{block}{Theorem}
		Any continuous and differentiable function $f$ with period $2\pi$ can be represented as
		\[
		f(t) = c + \sum_{k=1}^{\infty} a_k \cos(kt) + b_k \sin (kt)
		\]
	\end{block}
	
	\pause
	Practical recipe for daily data:
	\begin{itemize}[<+->]
		\item Add predictors $\cos\left(\frac{2\pi}{365} \cdot t\right)$ and $\sin\left(\frac{2\pi}{365} \cdot t\right) $
		\item Add predictors $\cos\left(\frac{2\pi}{365} \cdot 2t\right)$ and $\sin\left(\frac{2\pi}{365} \cdot 2t\right) $
		\item Add predictors $\cos\left(\frac{2\pi}{365} \cdot 3t\right)$ and $\sin\left(\frac{2\pi}{365} \cdot 3t\right) $
		\item \ldots
	\end{itemize}
	
	
	
	
\end{frame}

\begin{frame}{More predictors: Summary}
	
	\begin{itemize}[<+->]
		\item Use \alert{time} as a predictor
		\item Seasonality in predictors can be reflected using \alert{dummy variables} or
		using \alert{cosine} and \alert{sine} functions
	\end{itemize}
\end{frame}